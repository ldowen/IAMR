
\section{Initializing the Particles}

Particles are initialized from an ASCII file, identified in the \iamr inputs file:

\noindent {\bf particles.particle\_init\_file =}{\em particle\_file}

Here {\em particle\_file} is the user-specified name of the file.  The first line in this file is
assumed to contain the number of particles.  Each line after that contains the position of the particle as \\

x y z  \\

\section{Output Format}

\subsection{Checkpoint Files}

The particle positions and velocities are stored in a binary file in each checkpoint directory.  
This format is designed for being read by the code at restart rather than for diagnostics. \\

We note that the value of $a$ is also written in each checkpoint directory, 
in a separate ASCII file called {\em comoving\_a}, containing only the single value. \\

\subsection{Plot Files}

If {\bf particles.write\_in\_plotfile =} 1 in the inputs file 
then the particle positions and velocities will be written in a binary file in each plotfile directory.  

In addition, we can also
visualize the particle locations as represented on the grid.  The ``derived quantity''
{\bf particle\_count} represents the number of particles in a grid cell.
To add it to plotfiles, set \\

\noindent {\bf amr.derive\_plot\_vars = particle\_count} \\

\noindent in the inputs file

\subsection{ASCII Particle Files}

To generate an ASCII file containing the particle positions and velocities, 
one needs to restart from a checkpoint file from a run with particles, but one doesn't need to run any steps.
For example, if chk00350 exists, then one can set: \\

\noindent {\bf amr.restart = chk00350} \\
\noindent {\bf max\_step = 350} \\
\noindent {\bf particles.particle\_output\_file =} {\em particle\_output} \\

\noindent which would tell the code to restart from chk00350, not to take any further time steps, and to write an ASCII-format 
file called {\em particle\_output}. \\

\noindent This file has the same format as the ASCII input file: \\

\noindent number of particles \\ 
x y z  \\


\subsection{Run-time Screen Output}

The verbosity of the particle-related sections of code can be controled with

\noindent {\bf particles.pverbose } \\
